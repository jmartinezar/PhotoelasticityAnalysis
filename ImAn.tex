\documentclass{article}

\usepackage{graphicx}
\usepackage{amsmath}
\usepackage{amsfonts}
\usepackage[utf8]{inputenc}

\title{Photoelaticity}

\begin{document}

\maketitle

\section{Introduction}

\noindent Let us consider an electromagnetic wave with wavelength $\lambda$ and angular frequency $\omega$ propagating in the $\hat{z}$ direction. The electric field can be represented as
\begin{equation}
E(x,y,z;t) = \{E_{0x}e^{i\delta_x}, E_{0y}e^{i\delta_y}\}e^{i(2\pi z/\lambda - \omega t)}.
\end{equation}

\noindent The polarization state of the wave is determined by the complex amplitudes of its components. The evolution of the tip of the electric vector projected on the $z=0$ plane is given by
\begin{equation}
\frac{E_x^2}{|E_{0x}|^2} + \frac{E_y^2}{|E_{0y}|^2} - 2\frac{E_x}{|E_{0x}|}\frac{E_y}{|E_{0y}|}\cos(\Delta \delta) + \frac{E_y^2}{|E_{0y}|^2} = \sin^2(\Delta \delta),
\end{equation}
where $\Delta \delta = \delta_y - \delta_x$ is the phase difference between the two components of the electric vector. If

\begin{itemize}
    \item $\Delta \delta = m\pi$, with $m$ integer, the polarization state is linear.
    \item $\Delta \delta = (2m-1)\pi/2$, with $m$ integer, the polarization state is elliptical or circular. The axes of the ellipse are in the horizontal ($E_x$) and vertical ($E_y$) directions.
    \item $\Delta \delta$ has another value, the polarization state is a rotated ellipse. The rotation angle $\gamma$ of the major axis is given by
    \begin{equation}
    \tan 2\gamma = \tan 2\alpha \cos \Delta \delta,
    \end{equation}
    where $\tan \alpha = \left|\frac{E_{0y}}{E_{0x}}\right|$.
\end{itemize}

\section{Data and analysis}

For linearly polarized light passing through a phase-shifting sheet (such as an acrylic sheet), the light becomes elliptically polarized. To determine the polarization state parameters, such as the azimuthal angle $\gamma$ or the eccentricity $e$, we can place a linear polarizing filter at an angle $\theta$ relative to the horizontal and record the resulting light by capturing a photograph with a cellphone camera. The angle $\theta$ varies from 0 to 360 degrees, and the images are stored in the \textbf{data} directory. Data analysis allowed us to obtain the parameters that define the elliptical polarization state by fitting an elliptical function to the inscribed ellipse. A window of pixels is averaged to obtain data for a zone selected by ocular inspection of the images, with the center point located at pixel (510, 996). The result of the fit is shown in Figure \ref{1}.

\begin{figure}[!h]
  \centering
  \includegraphics[width=\linewidth]{1_polar_E.pdf}
  \caption{Plot of data and elliptical fit to data of 1 zone.}
  \label{1}
\end{figure}

Parameters values are shown in Table \ref{t1}

\begin{table}[h]
\centering
\begin{tabular}{|c|c|c|c|}
\hline
Color & R & G & B \\
\hline
$E_{\text{max}}$ & 189.54 & 206.5 & 198.59 \\
$E_{\text{min}}$ & 4.12 & 92.32 & 70.76 \\
$e$              & 0.999 & 0.894 & 0.934 \\
$\psi$           & $-23.35^\circ$ & $31.48^\circ$ & $60.39^\circ$ \\
\hline
\end{tabular}
\label{t1}
\caption{Polarization state parameters get of data analysis.}
\end{table}

\section{Conclusions}

Using a data analysis Python program is posible get polarization state parameters of elliptical polarized light that had been passing through a linear polarizing filter.

\end{document}
